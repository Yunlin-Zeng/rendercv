\documentclass[10pt, letterpaper]{article}

% Packages:
\usepackage[
    ignoreheadfoot, % set margins without considering header and footer
    top=1 cm, % seperation between body and page edge from the top
    bottom=1 cm, % seperation between body and page edge from the bottom
    left=1 cm, % seperation between body and page edge from the left
    right=1 cm, % seperation between body and page edge from the right
    footskip=0.5 cm, % seperation between body and footer
    % showframe % for debugging 
]{geometry} % for adjusting page geometry
\usepackage{titlesec} % for customizing section titles
\usepackage{tabularx} % for making tables with fixed width columns
\usepackage{array} % tabularx requires this
\usepackage[dvipsnames]{xcolor} % for coloring text
\definecolor{primaryColor}{RGB}{0, 0, 0} % define primary color
\usepackage{enumitem} % for customizing lists
\usepackage{fontawesome5} % for using icons
\usepackage{amsmath} % for math
\usepackage[
    pdftitle={Yunlin Zeng's CV},
    pdfauthor={Yunlin Zeng},
    pdfcreator={LaTeX with RenderCV},
    colorlinks=true,
    urlcolor=primaryColor
]{hyperref} % for links, metadata and bookmarks
\usepackage[pscoord]{eso-pic} % for floating text on the page
\usepackage{calc} % for calculating lengths
\usepackage{bookmark} % for bookmarks
\usepackage{lastpage} % for getting the total number of pages
\usepackage{changepage} % for one column entries (adjustwidth environment)
\usepackage{paracol} % for two and three column entries
\usepackage{ifthen} % for conditional statements
\usepackage{needspace} % for avoiding page brake right after the section title
\usepackage{iftex} % check if engine is pdflatex, xetex or luatex

% Ensure that generate pdf is machine readable/ATS parsable:
\ifPDFTeX
    \input{glyphtounicode}
    \pdfgentounicode=1
    \usepackage[T1]{fontenc}
    \usepackage[utf8]{inputenc}
    \usepackage{lmodern}
\fi

\usepackage{charter}

% Some settings:
\raggedright
\AtBeginEnvironment{adjustwidth}{\partopsep0pt} % remove space before adjustwidth environment
\pagestyle{empty} % no header or footer
\setcounter{secnumdepth}{0} % no section numbering
\setlength{\parindent}{0pt} % no indentation
\setlength{\topskip}{0pt} % no top skip
\setlength{\columnsep}{0.15cm} % set column seperation
\pagenumbering{gobble} % no page numbering

\titleformat{\section}{\needspace{4\baselineskip}\bfseries\large}{}{0pt}{}[\vspace{1pt}\titlerule]

\titlespacing{\section}{
    % left space:
    -1pt
}{
    % top space:
    0.3 cm
}{
    % bottom space:
    0.2 cm
} % section title spacing

\renewcommand\labelitemi{$\vcenter{\hbox{\small$\bullet$}}$} % custom bullet points
\newenvironment{highlights}{
    \begin{itemize}[
        topsep=0.10 cm,
        parsep=0.10 cm,
        partopsep=0pt,
        itemsep=0pt,
        leftmargin=0 cm + 10pt
    ]
}{
    \end{itemize}
} % new environment for highlights


\newenvironment{highlightsforbulletentries}{
    \begin{itemize}[
        topsep=0.10 cm,
        parsep=0.10 cm,
        partopsep=0pt,
        itemsep=0pt,
        leftmargin=10pt
    ]
}{
    \end{itemize}
} % new environment for highlights for bullet entries

\newenvironment{onecolentry}{
    \begin{adjustwidth}{
        0 cm + 0.00001 cm
    }{
        0 cm + 0.00001 cm
    }
}{
    \end{adjustwidth}
} % new environment for one column entries

\newenvironment{twocolentry}[2][]{
    \onecolentry
    \def\secondColumn{#2}
    \setcolumnwidth{\fill, 4.5 cm}
    \begin{paracol}{2}
}{
    \switchcolumn \raggedleft \secondColumn
    \end{paracol}
    \endonecolentry
} % new environment for two column entries

\newenvironment{threecolentry}[3][]{
    \onecolentry
    \def\thirdColumn{#3}
    \setcolumnwidth{, \fill, 4.5 cm}
    \begin{paracol}{3}
    {\raggedright #2} \switchcolumn
}{
    \switchcolumn \raggedleft \thirdColumn
    \end{paracol}
    \endonecolentry
} % new environment for three column entries

\newenvironment{header}{
    \setlength{\topsep}{0pt}\par\kern\topsep\centering\linespread{1.5}
}{
    \par\kern\topsep
} % new environment for the header

\newcommand{\placelastupdatedtext}{% \placetextbox{<horizontal pos>}{<vertical pos>}{<stuff>}
  \AddToShipoutPictureFG*{% Add <stuff> to current page foreground
    \put(
        \LenToUnit{\paperwidth-1 cm-0 cm+0.05cm},
        \LenToUnit{\paperheight-0.5 cm}
    ){\vtop{{\null}\makebox[0pt][c]{
        \small\color{gray}\textit{Last updated in July 2024}\hspace{\widthof{Last updated in July 2024}}
    }}}%
  }%
}%

% save the original href command in a new command:
\let\hrefWithoutArrow\href

% new command for external links:


\begin{document}
    \newcommand{\AND}{\unskip
        \cleaders\copy\ANDbox\hskip\wd\ANDbox
        \ignorespaces
    }
    \newsavebox\ANDbox
    \sbox\ANDbox{$|$}

    \begin{header}
        \fontsize{25 pt}{25 pt}\selectfont Yunlin Zeng

        \vspace{5 pt}

        \normalsize
        \mbox{\hrefWithoutArrow{mailto:yunlin_zeng@outlook.com}{yunlin\_zeng@outlook.com}}%
        \kern 5.0 pt%
        \AND%
        \kern 5.0 pt%
        \mbox{\hrefWithoutArrow{tel:+1-805-280-6729}{+1 805 280 6729}}%
    \end{header}

    \vspace{5 pt - 0.3 cm}


    \section{Education}



        
        \begin{twocolentry}{
            Expected Aug 2025
        }
            \textbf{Georgia Institute of Technology}, Ph.D. in Physics\end{twocolentry}



        \vspace{0.2 cm}

        \begin{twocolentry}{
            June 2019
        }
            \textbf{University of California, Santa Barbara}, B.S. in Physics\end{twocolentry}

        \vspace{0.10 cm}
        \begin{onecolentry}
            \begin{highlights}
                \item GPA: 3.90/4.00. Graduation with the highest honors; Highest academic honors for upper division physics courses; Dean's Honors x 6
            \end{highlights}
        \end{onecolentry}



    
    \section{Research Projects}



        
        \begin{twocolentry}{
            Sep 2023 – Now
        }
            \textbf{Graduate Researcher in Full-Waveform Variational Inference} \\ School of Computational Science and Engineering, Georgia Tech -- Atlanta, GA\end{twocolentry}

        \vspace{0.10 cm}
        \begin{onecolentry}
            \begin{highlights}
                \item Engineered an advanced variational inference framework with normalizing flows to efficiently solve the seismic inverse problem, enhancing the subsurface seismic imaging.
                \item Implemented robust uncertainty quantification methods to assess and interpret the stability and accuracy of seismic imaging predictions.
                \item Developed stochastic resampling and data augmentation techniques, and improve the generalization capabilities of initial migration velocity models in seismic analysis.
            \end{highlights}
        \end{onecolentry}


        \vspace{0.2 cm}

        \begin{twocolentry}{
            Jan 2020 – Dec 2021
        }
            \textbf{Graduate Researcher in Orbital Inference and Dynamics} \\ School of Physics, Georgia Tech -- Atlanta, GA\end{twocolentry}

        \vspace{0.10 cm}
        \begin{onecolentry}
            \begin{highlights}
                \item Applied Bayesian inference and parallel-tempering MCMC algorithms to constrain the orbital parameters of the Gliese 86 binary system, integrating diverse data types such as radial velocity and high-resolution imaging.
                \item Conducted simulations of stellar evolution within binary systems, reconstructed historical orbital dynamics, and contributed to theories of planet formation in truncated stellar disks.
                \item Investigated disk-satellite interactions within circumstellar disks to provide insights into planet formation dynamics under extreme conditions.
            \end{highlights}
        \end{onecolentry}


        \vspace{0.2 cm}

        \begin{twocolentry}{
            June 2019 – Dec 2019
        }
            \textbf{Software Developer and Research Analyst} \\ Physics Department, UC Santa Barbara -- Santa Barbara, CA\end{twocolentry}

        \vspace{0.10 cm}
        \begin{onecolentry}
            \begin{highlights}
                \item Developed 'orvara', an open-source Python software for Bayesian analysis of Keplerian orbits.
                \item Enhanced the computational efficiency by 5-10x over traditional methods and used low-level memory management to avoid python overheads.
                \item Applied MCMC methodologies to robustly sample posterior distributions of stellar and planetary orbits, ensuring high accuracy and reliability of model predictions.
                \item Authored several utility functions and extended the software’s capabilities to infer and visualize the results, broadening its applicability and user base.
            \end{highlights}
        \end{onecolentry}


        \vspace{0.2 cm}

        \begin{twocolentry}{
            June 2021 – June 2023
        }
            \textbf{Graduate Researcher in Computational Chemistry} \\ School of Physics, Georgia Tech -- Atlanta, GA\end{twocolentry}

        \vspace{0.10 cm}
        \begin{onecolentry}
            \begin{highlights}
                \item Led the design and implementation of a graphical user interface for the Force Field Toolkit (ffTK), streamlining the parameterization of small molecules based on quantum mechanical calculations.
                \item Integrated Psi4, an open-source quantum mechanics package, with ffTK, facilitating access to advanced computational tools for the scientific community.
                \item Enhanced the toolkit's functionality, including new command integrations and expanded input/output options, to support a wider range of quantum chemical computations.
            \end{highlights}
        \end{onecolentry}



    
    \section{Selected Publications}



        
        % \begin{samepage}
        %     \begin{onecolentry}
        %         \textit{Enhancing Full-Waveform Variational Inference through Stochastic Resampling and Data Augmentation}
        %     \end{onecolentry}

        %     \vspace{0.10 cm}
            
        %     \begin{onecolentry}
        %         \mbox{Yunlin Zeng}, \mbox{Rafael Orozco}, \mbox{Ziyi Yin}, \mbox{Felix J. Herrmann}

                
        % \end{onecolentry}
        % \end{samepage}

        \begin{samepage}
            \begin{onecolentry}
                \textbf{Y. Zeng}, R. Orozco, Z. Yin, and F. J. Herrmann, "Enhancing Full-Waveform Variational Inference through Stochastic Resampling and Data Augmentation," in \textit{International Meeting for Applied Geoscience and Energy}, Aug. 2024. 
            \end{onecolentry}
        \end{samepage}

        \begin{samepage}
            \begin{onecolentry}
                 \textbf{Y. Zeng}, T. D. Brandt, G. Li, T. J. Dupuy, Y. Li, G. M. Brandt, J. Farihi, J. Horner, R. A. Wittenmyer, R. P. Butler, C. G. Tinney, B. D. Carter, D. J. Wright, H. R. A. Jones, and S. J. O'Toole, "The Gliese 86 Binary System: A Warm Jupiter Formed in a Disk Truncated at ~2 au," \textit{The Astronomical Journal}, vol. 164, no. 5, p. 188, Nov. 2022.
            \end{onecolentry}
        \end{samepage}

        \begin{samepage}
            \begin{onecolentry}
                 T. D. Brandt, T. J. Dupuy, Y. Li, G. M. Brandt, \textbf{Y. Zeng}, D. Michalik, D. C. Bardalez Gagliuffi, and V. Raposo-Pulido, "orvara: An Efficient Code to Fit Orbits Using Radial Velocity, Absolute, and/or Relative Astrometry," \textit{The Astronomical Journal}, vol. 162, no. 5, p. 186, Oct. 2021.
            \end{onecolentry}
        \end{samepage}

        \begin{samepage}
            \begin{onecolentry}
                 \textbf{Y. Zeng}, A. Pavlova, P. M. Nelson, Z. L. Glick, L. Yang, Y. T. Pang, M. Spivak, G. Licari, E. Tajkhorshid, C. D. Sherrill, and J. C. Gumbart, "Broadening Access to Small-Molecule Parameterization with the Force Field Toolkit," \textit{The Journal of Chemical Physics}, vol. 160, no. 24, p. 242501, Jun. 2024.
            \end{onecolentry}
        \end{samepage}


    
    \section{Technologies}



        
        \begin{onecolentry}
            \textbf{Languages:} Python, Julia, Cython, TCL, Fortran, Mathematica, MATLAB
        \end{onecolentry}

        \vspace{0.2 cm}

        \begin{onecolentry}
            \textbf{Technologies:} Applied Machine Learning (Scikit-learn), Deep learning (PyTorch, TensorFlow), Signal Processing
        \end{onecolentry}


    
    


\end{document}